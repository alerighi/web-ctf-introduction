\documentclass{beamer}
\usepackage[italian]{babel}
\usetheme{Madrid}

\title{Introduzione alle CTF}
\subtitle{Lezione 1}
\author{Alessandro Righi \and Cristiano Di Bari}
\institute{Università degli Studi di Verona}
\date{3 Novembre 2023}

\begin{document}
\begin{frame}
\titlepage
\end{frame}

\begin{frame}{Outline}
\tableofcontents
\end{frame}

\section{Introduzione}

\begin{frame}{Cosa sono le CTF?}

Le \textit{Capture The Flag} (CTF) sono delle sfide in cui i partecipanti 
devono trovare delle \textit{flag}, ossia delle stringhe di testo, all'interno
di sistemi informatici contenenti delle vulnerabilità di sicurezza. 
\end{frame}

\begin{frame}{Tipologie di CTF}
Esistono due tipologie di CTF:
\begin{itemize}
    \item \textit{jeopardy}: il partecipante attacca una serie di servizi malevoli e sottomette le flag ad un sistema di verifica
    \item \textit{Attack-Defence (AD)}: competizione a squadre dove ogni team deve attaccare i sistemi dell'avversario, e difendere i propri
\end{itemize}

Per queste lezioni di concentreremo sul primo tipo, di gran lunga le più diffuse, che è anche
quello che organizza \textit{Wurth}.
\end{frame}

\begin{frame}{Tipologie di challenge}
Tipicamente le challenge che si affrontano possono essere di 4 macro categorie:
\begin{itemize}
    \item \textit{Binary} è necessario ricercare vulnerabilità in un eseguibile, quali ad es. buffer overflow
    \item \textit{Web} si tratta di trovare vulnerabilità in una web app, web API, o comunque applicativo esposto in rete
    \item \textit{Crypto} è necessario decodificare un testo cifrato con un algoritmo (ovviamente vulnerabile)
    \item \textit{Misc} sono challenge che non rientrano in nessuno dei tipi precedenti, e richiedono spesso creatività per essere affrontate
\end{itemize}

Per queste lezioni ci concentreremo sulla categoria \textit{Web}.
\end{frame}

\section{Path traversal}
\begin{frame}{}
\end{frame}

\subsection{Tool: Burp suite}

\begin{frame}{}
\end{frame}

\section{SQL Injection}
\begin{frame}{}
\end{frame}

\subsection{Tool: sqlmap}
\begin{frame}{}
\end{frame}


\section{Cross site scripting}
\begin{frame}{}
\end{frame}

\subsection{Tool: ngrok}
\begin{frame}{}
\end{frame}

\end{document}
