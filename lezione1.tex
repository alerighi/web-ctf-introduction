\documentclass[12pt]{beamer}
\usepackage{tikz}
\usepackage[italian]{babel}
\usetheme{Madrid}
\usecolortheme{orchid}

\title{Introduzione alle CTF}
\subtitle{Lezione 1}
\author{Alessandro Righi \and Cristiano Di Bari}
\institute{Università degli Studi di Verona}
\date{3 Novembre 2023}

\begin{document}
\begin{frame}
\titlepage
\end{frame}

\section{Introduzione}

\begin{frame}{Chi siamo?}
\begin{columns}[T] % align columns
\begin{column}{.48\textwidth}
\begin{itemize}
\item laurea magistrale @ UniVR
\item istruttore @ CyberChallenge
\item \textit{System Developer} @ IOTINGA
\end{itemize}        
{\color{blue}\rule{\linewidth}{2pt}}%
\begin{center}
Cristiano Di Bari
\begin{tikzpicture}[inner ysep=0.4cm]
\clip (0,0) circle (1.5cm) node {\includegraphics[width=3cm]{img/ale.jpeg}};
\end{tikzpicture}
\end{center}
\end{column}%
\hfill%
\begin{column}{.48\textwidth}%
\begin{center}%
\begin{tikzpicture}%
\clip (0,0) circle (1.5cm) node {\includegraphics[width=3cm]{img/ale.jpeg}};
\end{tikzpicture}

Alessandro Righi
\color{blue}{\rule{\linewidth}{2pt}}%
\begin{itemize}
\item laurea magistrale @ UniVR
\item istruttore @ CyberChallenge
\item \textit{Applied Research Director} @ IOTINGA
\end{itemize}
\end{center}
\end{column}%
\end{columns}
\end{frame}

\begin{frame}{Cosa sono le CTF?}

Le \textit{Capture The Flag} (CTF) sono delle sfide in cui i partecipanti 
devono trovare delle \textit{flag}, ossia delle stringhe di testo, all'interno
di sistemi informatici contenenti delle vulnerabilità di sicurezza. 
\vfill
\begin{exampleblock}{Esempio di flag}
\texttt{CCIT\{Th1s-1sY0uR-F1rst-Fl4ag\}}
\end{exampleblock}
\vfill
Una volta trovate le flag vanno, solitamente, inviate ad una piattaforma di gara, 
che si occupa di validarle, assegnando il punteggio della challenge nel caso siano corrette.
\end{frame}

\begin{frame}{Perché fare CTF?}
Avete mai fatto CTF?

\pause\vfill

Se no... ecco alcune ragioni per iniziare:
\pause\vfill
\begin{itemize}
\item per divertirsi!
\pause\vfill
\item per imparare cose nuove (tante)
\pause\vfill
\item per scrivere software (più) sicuro
\pause\vfill
\item per conoscere nuova gente, creare networking
\end{itemize}
\end{frame}

\begin{frame}{Tipologie di CTF}
Esistono due tipologie di CTF:
\begin{itemize}
    \item \textit{Jeopardy}: il partecipante attacca una serie di servizi malevoli e sottomette le flag ad un sistema di verifica
    \item \textit{Attack-Defence (AD)}: competizione a squadre dove ogni team deve attaccare i sistemi dell'avversario, e difendere i propri
\end{itemize}

\vfill
Per queste lezioni di concentreremo sul primo tipo (\textit{Jeopardy}), di gran lunga le più diffuse, che è anche
quella organizzata da \textit{Würth Phoenix}.
\end{frame}

\begin{frame}{Tipologie di challenge}
Tipicamente le challenge che si affrontano possono essere di 4 macro categorie:
\begin{itemize}
    \item \textit{Binary}: è necessario ricercare vulnerabilità in un eseguibile, quali ad es. buffer overflow
    \item \textit{Web}: si tratta di trovare vulnerabilità in una web app, web API, o comunque applicativo esposto in rete
    \item \textit{Crypto}: è necessario decodificare un testo cifrato con un algoritmo (ovviamente vulnerabile)
    \item \textit{Misc}: sono challenge che non rientrano in nessuno dei tipi precedenti, e richiedono spesso creatività per essere affrontate
\end{itemize}

Per queste lezioni ci concentreremo sulla categoria \textit{Web}.
\end{frame}

\begin{frame}
\Huge\center Iniziamo!
\end{frame}

\section{Path traversal}
\subsection{Introduzione}
\begin{frame}{Path traversal}

Immaginiamo di avere una pagina web che per caricare l'immagine del profilo di un utente effettua una richiesta a:

\begin{block}{Richiesta}
\texttt{https://mysecureapp.com/assets?name=image.jpeg}
\end{block}

\pause
Cosa succede se modifico la richiesta in questo modo?

\begin{exampleblock}{Richiesta alterata}
\texttt{https://mysecureapp.com/assets?name=../image.jpeg}
\end{exampleblock}
    
\pause

Se il server non effettua adeguati controlli, è possibile leggere file fuori dalla \textit{root} directory del web server!

\end{frame}
\begin{frame}{Path traversal}
Cosa consente di fare questa vulnerabilità?

\pause
\begin{itemize}
    \item leggere \textit{segreti} altrimenti non accessibili, ad es. file di configurazione quali \texttt{/etc/passwd}
    \pause
    \item ottenere il \textit{codice sorgente} dell'applicazione web
    \pause
    \item accedere ai dati di altri utenti, bypassando restrizioni imposte dall'applicazione web
    \pause
\end{itemize}

\begin{alertblock}{Suggerimento}
È possibile aggiungere tanti \texttt{../} fino a raggiungere la directory \textit{root}, ad esempio \texttt{../../../../../etc/passwd}
\end{alertblock}

\end{frame}
\subsection{Challenge: }
\begin{frame}{Challenge}
Vediamo la prima challenge. Per queste lezioni utilizzeremo delle challenge 
prese dalla piattaforma di allenamento delle Olimpiadi di Cybersecurity (\url{https://olicyber.it}), 
a cui vi invitiamo ad iscrivevi.
\end{frame}

\subsection{Tool: Burp suite}

\begin{frame}{}
\end{frame}

\section{SQL Injection}
\begin{frame}{}
\end{frame}

\subsection{Tool: sqlmap}
\begin{frame}{}
\end{frame}


\section{Cross site scripting}
\begin{frame}{}
\end{frame}

\subsection{Tool: ngrok}
\begin{frame}{}
\end{frame}

\end{document}
